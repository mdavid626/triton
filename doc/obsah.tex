%==============================================================================
% (c) David Molnar 2015
%==============================================================================

\chapter{Úvod}
Vyvíjet software v dnešní době není jednoduché. Konkurence na trzích je silná, požadavky na software se mění velice rychle a často je potřeba vydávat novou verzi aplikace co nejrychleji.

Za základní zásadu pro vývojový tým můžeme považovat Continuous Integration. Pomocí tohoto konceptu tým zůstane v synchronizaci a dokáže odstranit zpoždění způsobené integračními chybami. Je důležité si ale uvědomit, že Continous Integration je pouze prvním krokem k dosažení cíle. Další stanice je Continuous Delivery, tzn. nejenom častá integrace, ale i časté nasazení a uvolnění nové verze aplikace do produkce. Rozchodit nasazení ve smyslu Continuous Delivery není jednoduché, záleží to samozřejmě i na složitosti projektu. Nesmíme se ale zapomenout na okamžité výhody. Dlouhé, pracné a problematické vydávání verzí a jejich nasazení se stane věcí minulosti. Díky tomu zákazníci uvidí okamžitý pokrok vývoje objednaného produktu, který dodá funkcionalitu, kterou potřebují a využívají každý den. Je potřeba se zmínit i o výhodě, že tímto se také odstraní jeden u největších zdrojů stresu a napětí v týmu.

Velice inspirativní je projekt a tým vedený panem Kentem Beckem \cite{ContDelivery}. Tento velice disciplinovaný tým každý večer nasazoval novou verzi své aplikace přímo do produkce. Takový způsob implementace má několik výhod: napsaný a hotový kód neleží ve verzovacím systému\footnote{SVN, Git, TFS atd.} bez využití a tým dokáže velice rychle reagovat na problémy a na nové příležitosti. Navíc je zajímavý, že tento způsob práce vedlo k prohloubení vztahů mezi členy týmu a nárůstu důvěry zákazníka ve firmě. 

Continous Delivery dokáže snížit dobu trvání cyklu od první myšlenky a nápadu až do použitelný software. Myšlení ve smyslu Continuous Delivery upadlo na dlouhou dobu v zapomenutí, někde mezi vývojovým a nasazovacím týmem. Proto je velice důležité tyto dva týmy co nejvíce spojit. Nesmíme se zapomenout na to, že základem všeho je vysoká stupeň automatizace -- jednotlivé operace budou rychlé, opakovatelné, snadno testovatelné a kvalitní (bez chyb).

Našim hlavními průvodci v této práci budou Jez Humble a David Farley, průkopníci a známí aktivisti zásad Continous Delivery. Jejich kniha \cite{ContDelivery} bude pro nás sloužit jako Bible. 

Tato práce se detailně zabývá implementací zásad Continuous Delivery při urychlení vývoje a nasazení jedné ukázkové aplikace. Ukázkový projekt je webová aplikace s MSSQL databází. Dále obsahuje i jiné části, které budou popsané v kapitole \ref{ch:impl}. 

V kapitole \ref{ch:teorie} bude čtenář seznámen se základními pojmy z oblasti agilních metodik vývoje a doručení softwarových produktů. Kapitola \ref{ch:impl} se detailně zabývá návrhem a implementací ukázkové aplikace. Vzhledem k tomu, že je to ukázková aplikace, nebudou všechny detaily implementovány. V pokračování je popsán vývoj jednotlivých nástrojů a vytvoření infrastruktury (Active Directory domain, testovací prostředí atd.), kam bude aplikace nasazena. Vyhodnocení výsledků se uskuteční v kapitole \ref{ch:vyslekdy}. Pokusíme se porovnat výhody a nevýhody bez aplikování a pomocí Continous Delivery. V závěru v kapitole \ref{ch:zaver} bude následovat souhrn práce.

Naším cílem tedy bude ukázat, že je možný dodat vysoce kvalitní softwarový produkt velice často, třeba i denně a to bez zbytečných problémů a zádrhelů. Budeme k tomu používat hlavně Microsoft technologie.

V celé této práci budeme označovat pojmy Continous Delivery zkratkou CD a Continous Integration zkratkou CI. Ostatní zkratky jsou v tabulce zkratek na konci práce.

\chapter{Teoretická část}
\label{ch:teorie}
V této kapitole seznámíme čtenáře se základy Continous Delivery. Ukážeme, jaké problémy lze vyřešit pomocí tohoto konceptu. Budeme čerpat z knihy\cite{ContDelivery}.

\section{Deploymnet pipeline}
Ve vývoji software je velice důležitý, jakým způsobem bude vyvíjený produkt dodán uživateli. Jaká je rychlost, efektivita, spolehlivost a kvalita dodání. Většina vývojářů se soustředí na analýzu, psaní kódu a někteří i na testování (unit testy, integrační testy atd.).
Existuje spousta metodik pro vývoj software, většina z nich se ale  soustředí na zpracování požadavku a na jeho vliv na proces vývoje.

Nejdůležitější technikou pro nás bude tzv. deployment pipeline. Deployment pipeline je automatizovaná implementace procesu buildování, testování, nasazení naší aplikace. Samozřejmě každá organizace bude mít svoji vlastní implementaci, nicméně principy a zásady budou stejné. Příklad deployment pipeline-u je vidět na obrázku \ref{fig:pipeline}.

\begin{figure}[]
  \centering
  \includegraphics[width=15cm]{fig/pipeline_overview.png}
  \caption{Přehled deployment pipeline}
  \label{fig:pipeline_overview}
\end{figure}

\begin{figure}[]
  \centering
  \includegraphics[height=12cm]{fig/pipeline_basic.png}
  \caption{Základní deployment pipeline}
  \label{fig:pipeline}
\end{figure}

\begin{figure}[]
  \centering
  \includegraphics[width=15cm]{fig/pipeline_changes.png}
  \caption{Změny přecházející se přes deployment pipeline}
  \label{fig:pipeline_overview}
\end{figure}

Každá změna ve zdrojovém kódu, v konfiguraci nebo v prostředí způsobuje vytvoření nové instance pipeline-u. První krok je vytvoření binárek a installerů. Pak následuje testování, které ověří, jestli produkt má dostateční kvalitu. Každý test nám poskytne důvěru, že daná kombinace binárek, konfigurace a prostředí bude fungovat. Testování nemusí být vždy plně automatizovaný -- často je potřeba, aby člověk manuálně ověřoval výstupy. 

Největší výhodou deployment pipeline-u je, že zviditelní každou část vývojového procesu pro všechny členové týmu. Tým takto dokáže zajistit lepší kolaboraci. Další výhodou je i zlepšení zpětné vazby -- problémy jsou identifikovány a odstraněny co nejrychleji. Umožňuje i to, aby tým mohl nasazovat jakoukoli verzi softwaru kdykoli pomocí zcela automatizovaného procesu.

\section{Běžné antipatterny nasazování}
Krok nasazení u většiny moderních aplikací je procesem komplexním, zahrnujícím spoustu dílčích kroků. Mnoho organizací nasazuje software manuálně. Každý krok se provede manuálně, jako atomická operace. Všechna rozhodnutí, dělaná během nasazení, jsou náchylná k lidským chybám. Navíc jednotlivé kroky je možné udělat po každém trošku jinak, co může vést k různým nejasným a nekvalitním výsledkům.

\subsection{Manuální nasazování}
Tento bod může být charakterizován následovně:

\begin{itemize}
  \item Existuje velice detailní dokumentace, který popisuje proces nasazení
  \item Při ověření kvality se spoléhá především na manuální testování 
  \item Příliš časté dotazy na vývojový tým, ohledně vysvětlení chyb při nasazení
  \item Nasazovací proces se v průběhu vydání verze často modifikuje a opravuje
  \item Prostředí mají rozdílné konfigurace, např. aplikační servery s různými konfiguracemi aplikačního poolu, nekonzistentní struktury složek, různá verze prerekvizit atd.
  \item Nasazení a vydání verze trvá hodiny, občas i několik dnů
\end{itemize}

\subsection{Nasazení do produkce až po dokončení vývoje}
V tomto případě je aplikace nasazena do produkce až ve chvíli, kdy vývojový tým zcela dokončil vývoj. Charakterizace:

\begin{itemize}
  \item Jestliže aplikace byla testována, testeři testovali na vývojových stanicích
  \item Lidi z nasazovacího týmu uvidí novou verzi aplikace teprve během nasazení. V některých organizacích existují dva týmy: jeden pro nasazení do testovacího prostředí a jeden pro produkci. Vybudovat prostředí, které je podobné produkci, je velice drahé; přístup k němu je přísně kontrolované, případně je možné, že vůbec neexistuje.
  \item Vývojový tým sestaví binárky, instalátory, konfigurační soubory, databázové migrační skripty a nasazovací dokumentaci, které pak předá nasazovacímu týmu. Tento tým následně provede samotné nasazení -- většinou do prostředí, které předtím nebylo nikdy otestováno.  
  \item Kolaborace mezi vývojovými a nasazovacími týmy je slabé, nebo vůbec neexistuje. Lidé nechtějí zlepšit komunikaci, raději se spoléhají na dokumentaci atd.
\end{itemize}

Slabá spolupráce je pak kompenzována ad-hoc řešením: emaily, telefonní hovory, rychlé opravy. Přináší to stres, spěch a nízkou kvalitu. Disciplinovaný tým zahrne všechny informace o nasazení do nasazovacího plánu, ale i přesto je tento postup málokdy účinný. Většinou je potřeba dodat produkt v určitém termínu, což může vést k nárůstu tlaku mezi členy týmu a ke zhroucení předem definovaného postupu nasazení. 

V následujících kapitolách ukážeme, jak nám může pomoct integrování testovacích, nasazovacích činností do vývojového procesu. Budou to činnosti každodenní práce. Když přijde čas nasazení do produkce, riziko, že něco nefunguje, bude velice malé -- postupy již budou ověřeny a vyzkoušeny týmem mnohokrát v nejrůznějších prostředích. K tomu ovšem musí být zajištěna časná spolupráce všech lidí, podílejících se na vývoji a doručení.

\subsection{Manuální správa konfigurace produktivního prostředí}
Mnoho organizací spravuje konfigurace produktivních prostředí prostřednictvím speciálního týmu -- tzv. nasazovacího týmu. Pokud je potřeba udělat nějakou změnu, např. upravit parametry databázového připojení v connection stringu, zvýšit maximální povolenou velikost příchozího requestu na webovém serveru, pak potřebné akce provede tento tým zcela manuálně, přímo v produkci. Ve většině případů historie změn neexistuje, pokud ale ano, je to manuální záznam v databázi, na kterou se často zapomene.

\begin{itemize}
  \item Nasazení do testovacího prostředí bylo několikrát úspěšné, nasazení do produkce se přesto nezdaří.
  \item Nasazovací tým dlouhou dobu připravuje prostředí pro nasazení.
  \item Nelze udělat krok zpátky na dřívější konfiguraci systémů, mezi které patří operační systémy, aplikační, webové a databázové servery a jiné infrastrukturální nastavení.
  \item Produkční a jiné servery mají různá verze operačních systémů, aktualizací, knihoven a to zcela neúmyslně.
  \item Úprava konfigurace je provedena přímo na produkčních serverech, např. prostřednictvím RDP, SSH atd.
\end{itemize}

Všechny aspekty testovacích, produkčních a jiných prostředí, obzvlášť konfigurace aplikací třetích stran, by měli být uplatněny z verzovacího systému pomocí automatizovaného procesu. Klíčová věc je správa konfigurace, např. i to, že jsme schopni obnovit každou část infrastruktury používanou naší aplikací, tzn. operační systémy, aktualizace, konfigurace OS, knihovny třetích stran atd. 

Nejjednodušším způsobem, jak docílit automatizované konfiguraci, je virtualizace. Nemělo by docházet k manuálnímu provedení změn v testovacích ani v produkčních prostředích. Jediným způsobem by mělo být vytvořený automatizovaného procesu.

Vyvíjená aplikace často závisí na jiné aplikaci. Z toho důvodu je důležité, aby bylo možné rychle zobrazit verze aktuálně nasazované aplikace, operačního systému a aplikací třetích stran. Během většiny nasazení se dělají změny až po poslední chvíli. Musí existovat takový způsob zavádění změn, který je zaznamenáván a otestován. Tyto změny by pak měly být propagovány prostřednictvím automatizovaného procesu. V případě, že se nasazení nezdaří, mělo by být možné provedené změny nebo předchozí verzi aplikace vrátit.

\section{Jak dosáhnout cíle?}
Prvním klíčovým slovem je \textbf{automatizace}. V případě, že proces nasazení, testování a buildování není automatizován, pak není ani opakovatelný. Po každém se bude lišit kvůli změnám v konfiguraci aplikace, konfiguraci operačního systému a prostředí. 
Manuální kroky jsou náchylné k chybám a navíc není možné zpětně určit, co přesně bylo provedeno. Takovým způsobem nelze zaručit vysokou kvalitu výsledků. Vydání nové verze se velice často stává uměním. 

Další klíčovou vlastností je, aby proces byl prováděn \textbf{často}. Pokud nové verze vydáváme často, pak rozdíly mezi nimi budou malé. Pro nás to znamená, že nasazení lze snadno vrátit -- riziko je malé, protože jsme provedli málo změn. Časté nasazení vede i ke zlepšení zpětné vazby od zákazníka. Jeden z našich hlavních cílů je, dostat zpětnou vazbu co nejdříve. Kód, konfigurace, nastavení a funkcionalita, které nikdo předtím neviděl a jenom leží ve verzovacím systému, je rizikové -- může se nám zdát, že je vše v pořádku a překvapení přijde až po dlouho dobu, často ve formě supportních ticketů a nespokojených zákazníků.

\section{Hlavní výhody CD}
Hlavní výhodou zásad CD, které jsme popisovali v předchozí části je to, že vytváří proces, který je opakovatelný, spolehlivý a předvídatelný. Tyto vlastnosti vedou ke snížení doby cyklu vývoje, s tím naši uživatelé dostanou novou funkcionalitu a opravy chyb mnohem rychleji. Musíme si uvědomit, že může to znamenat nějakou investici, kterou ale rychle získáme zpátky.

\subsection{Posílení týmu}
Jednou z klíčových zásad CD je, že je to systém, který umožní testerům, nasazovacího týmu a lidem z podpory, aby aplikaci sami zprovoznili v libovolné verzi, do libovolného prostředí. Praxe ukázala, že doba trvání vývojového cyklu je ovlivněna i tím, že členové týmu čekají na nějakou dobrou verzi aplikace. Často to přináší nekonečnou emailovou komunikaci, založení supportních ticketů atd. Pomocí deployment pipelinu tento problém je zcela odstraněn. Každý má možnost vidět dostupné verze a po vybrání aplikace nasadit pouhým kliknutím na ikonku.

\subsection{Snížení počtu chyb}
Do dané aplikace se mohou dostat chyby z nejrůznějších míst. Uživatelé mohou požádat o nevhodnou věc. Analytik zachycující požadavek, pochopí to spatně. Vývojáři vytvoří ne úplně správně fungující až chybný kód. 
Aby naše aplikace fungovala, je potřeba zajistit spoustu věcí: správnou verzi zdrojového kódu, správnou verzi databázového schématu, správnou konfiguraci webového serveru, ale musí být nastaven třeba i správný URL na nějaký externí systém. Správa konfigurací znamená ovládání těchto informací, od prvního bitu až po poslední.

Nechme počítačům práci, v čem jsou opravdu dobré: správa zdrojového kódu, konfiguračních souborů, detekování změn, skripty pro vytvoření databází, schémata, konfigurace prostředí, operačních systémů atd. Takovým způsobem jsme schopni zajistit, aby všechno probíhalo tak, jak to daná aplikace vyžaduje.

\subsection{Snížení stresu}
Možná největší výhodou CD je snížení stresu všech stran, zúčastněných na vydání nové verze aplikace. Většina lidí, kteří již zažili blížící se termín odevzdání/zveřejnění aplikace mohou potvrdit, jak moc stresující tato zkutečnost může být. Samotný stres může znamenat zdroj problémů v našem procesu vývoje. Může vést ke vzniku problémů. Úpravy např. databáze, databázových schémat přímo na produkčních serverech jenom proto, abychom zprovoznili aplikaci, nejsou dobrou praxí. 
Tlak, který způsobuje takový stres, vede k použití rychlých hacků. Problém jako takový je to, že vydání nové verze aplikace není běžnou činností, ale velkou událostí.

\subsection{Flexibilita nasazení}
Nastartovat naši aplikaci v novém prostředí by mělo být jednoduchým úkolem. Ideálně to znamená vytvoření virtuálních strojů (např. ze šablony) a pak vytvoření konfigurace, která je unikátní pro toto prostředí. Dále můžeme použít náš automatizovaný proces. V prvním kroku připravit prerekvizity, v druhým nasadit vybranou verzi aplikace.

\subsection{Cvičení dělá mistra}
Můžeme říci, že každý tým, který používá CI nebo iterativní vývojové techniky, bude potřebovat nasadit aplikaci často. Nejlepší strategií je, používat stejné postupy, instalátory atd., které budeme používat v produkčních prostředích. Nesmíme mít rozlišnou strategii pro testovací a produkční prostředí, anebo speciální tým a postup. 

Takovým způsobem každý den ověříme, že naše aplikace a nasazovací proces funguje. Jediným případem, kdy můžeme udělat výjimky, jsou vývojové stanice. Vývojáři mohou potřebovat např. sestavit svoje vlastní binárky. Snažme se ale i v tomto případě použít co nejvíce procesů, které používáme v produkci.

\section{Zásady doručení software-u}
V této sekci shrneme nejvýznamnější zásady, beze kterých žádný nasazovací proces nemůže efektivně fungovati. 

\subsection{Opakovatelné a spolehlivé nasazení}
Vydání nové verze aplikace by mělo být snadné, jelikož jsme nasazení vyzkoušeli a otestovali miliónkrát předtím. Opakovatelnost a spolehlivost vycházejí ze dvou zásad: automatizovat téměř vše a udržovat zdrojové soubory, konfigurace, skripty atd. ve verzovacím systému.

Nasazení aplikace může být zcela automatizované. Konfigurace aplikace taky může být automatizované, všechny potřebné informace uložené ve verzovacím systému. Je jasné, že fyzický hardware nemůže být ve verzovacím systému, ale virtualizace může významně pomoct.

\subsection{Automatizování téměř všeho}
Určitě existují věci, které není možné automatizovat. Jako příklady se dají uvést explorativní testování, kontrola některých výstupů a schválení nasazení. Některé typy nasazení, např. nasazení do produkčních prostředí, mohou vyžadovat schválení vedoucího. Musíme si ale uvědomit, že automatizace je možné ve více případech, než si na první pohled myslíme. Existují případy, kdy automatizace je extrémně náročná. Pro tyto případy je dobrou radou, že je lze vyřešit až na konci, nebo nevyřešit vůbec a najít jinou alternativní řešení. 

\subsection{Všechno je ve verzovacím systému}
Všechno, co je potřeba k sestavení, nasazení a testovaní aplikace, musí být ve verzovacím systému. Sem patří i specifikační dokumenty, testovací skripty, automatizované test cases, skripty ke konfiguraci, databázové skripty, knihovny, různé nástroje, technické dokumentace atd.
Je potřeba umožnit pro komukoliv, kdo se sedne k jedné ze stanic, aby mohl stáhnout aktuální verzi aplikace, následně provedl build a pak nasazení aplikace do libovolného prostředí. Dále je potřeba, aby aktuálně nasazená verzi bylo možné jednoduše identifikovat.

\subsection{Provádět často}
Tato zásada je z výše uvedených zásad nejobecnějším. Můžeme ji považovat spíš jako heuristiku. Integrace různých části velké aplikace je problematické, je to bolestivým procesem. Proto je důležité integrovat každý jediný commit/změnu ve verzovacím systému. Pokud např. otestování nové verze aplikace je časově náročné a neefektivním procesem, musíme to dělat častěji, třeba i každý den. To bude mít následek, že tým postupně vylepší daný proces a odstraní problémy.

\subsection{Zabudování kvality}
Opravit chybu v produkční aplikaci je velice drahé. Opravit tu stejnou chybu v testovacím prostředí je mnohem levnější. V ideálním případě by chyby měli být zachyceny ještě před komitnutím do verzovacího systému. Proto je důležité aplikovat ty zásady, které popisujeme. Pomohou totiž zachytit chyby co nejdříve.

Testování nesmí být považováno jako fáze následující po dokončení vývoje. Pokud testování necháme až na konec, tak to je již pozdě. Nebude dostatek času na opravování chyb a odstranění problémů. Navíc, uvědomujme si, že testování není doména jenom testerů, ale za kvalitu ručí celý tým jako celek, tzn. každý člen týmu.

\subsection{Hotové znamená nasazeno}
Pojem \uv{hotové} pro většinu vývojových týmu znamená, že daná funkcionalita je vyvinutá. Neznamená to ovšem, že zákazník funkcionalitu již viděl a již vůbec ne to, že nová verze aplikace s funkcionalitou je již nasazena. 

Nejlepší je, pokud pojem \uv{hotové} znamená následující: funkcionalita je hotová, pokud byla uživatelům úspěšně demonstrována v takovém prostředí, které je podobné produkčnímu. 

\subsection{Každý je zodpovědný za nasazení}
V ideálním případě, každý člen týmu vykonává svoji práci v prospěch týmu. Tým nakonec dosáhne úspěch nebo selže společně, jako tým a ne jednotlivce. Bohužel ale realita ve většině případu je jiná. Vývojáři hážou svoji hotovou práci na testeři, testeři dále na nasazovací tým. Pokud něco nefunguje, selhává, pak tým tráví víc času obviňováním, než opravou. Proto je velice důležité odstranit bariéry mezi těmito týmy. Někdy může stačit reorganizovat kancelář, aby tyto týmu byly blíž k sobě.

\subsection{Neustálé zlepšování}
Je potřeba zdůrazňovat, že nasazovací proces není statický, vyvíjí se spolu s aplikací. Tým by měl regulárně revidovat nasazovací proces a prodiskutovat, jaké zlepšení se v následující době uplatní.

\section{Slovníček pojmů}
Nasleduje vysvětlení vybraných pojmů používané v této práci:

\paragraph{Source code} zdrojový kód
\paragraph{Version control} správa zdrojových souborů, např. Git, TFS, SVN
\paragraph{Deploy, deployment} nasazení (instalace) a konfigurace aplikace
\paragraph{UAT: User Acceptance Testing} uživatelské akceptační testy
\paragraph{Production} produkce, produkční prostředí (PP)
\paragraph{Smoke test} jednoduchý test, který ověří, že aplikace běží
\paragraph{Artifact repository} místo, kde jsou umístěny binárky (artefakty) aplikace
\paragraph{Operations, delivery team} tým, který provádí nasazení a údržbu aplikace
\paragraph{CI: Continuous Integration} častá integrace dílčích části softwarového produktu
\paragraph{CD: Continuous Delivery} rozšíření CI o nasazení
\paragraph{Continuous Deployment} rozšíření CD, nasazení do PP je prováděno automaticky

\section{Technologie a nástroje}
V této sekci budou popsány vybrané nástroje a technologie, které budeme používat v dalších sekcích a pomohou nám dosáhnout CD.

\subsection{Windows Server 2012 R2}
Windows Server je serverový operační systém společnosti Microsoft. Podporuje různé služby a funkce, např. Active Directory, DNS Server, DHCP Server, Group Policy, IIS a jiné. Nejnovější verze k dnešnímu datu (první čtvrtletí roku 2016) je Windows Server 2012 R2. Tuto verzi budeme používat i my. \cite{MasteringWindowsServer}

Window Server 2012 R2 podporuje dvě módy instalace: Server Core a Server with a GUI. Server Core neobsahuje GUI, jenom rozhraní s příkazovým řádkem. Minimální nároky na hardware jsou 1,4 GHz x64 procesor, 512 MB paměť a 32 GB místo na disku. Jsou dostupny čtyři různé edice (viz \cite{WindowsServer2012Editions} a \cite{WindowsServer2012Licensing}):

\begin{itemize}
\item Datacenter: nejvyšší edice, bez licenčních omezení, virtualizační práva nejsou omezena
\item Standard: to samý jako Datacenter, kromě virtualizačních práv (jsou povoleny dvě virtualizované instance)
\item Essentials: max. 25 uživatelů, omezený hardware, povolena jedna virtualizovaná instance, chybí: možnost připojení do domény, Terminal Services Gateway, Data Deduplication, Failover Clustering, Server Core mode, Volume Activation Services a jiné.
\item Foundation: podobný edice Essentials, max. 15 uživatelů, některé funkce nejsou dostupny
\end{itemize}

Dále existují i speciální produkty, které vychází z Windows Server 2012 R2, jako např. Microsoft Hyper-V Server 2012 R2, Windows Storage Server 2012 R2 Standard a Windows Storage Server 2012 R2 Workgroup.

Důležité služby a funkce operačního systémy jsou následovní:

\paragraph{Active Directory}
Implementace adresářových služeb LDAP. Umožňuje mimo jiné: nastavování politik a instalace programů hromedně na více počítačů. Často používaným pojmem je \uv{doména}: je to skupina počítačů sdílejících společnou adresářovou databázi. Ještě větší skupina je \uv{les}. Server, na který je nainstalována role Active Directory Domain Services, je nazýván tzv. \uv{domain controller}. \cite{MasteringWindowsServer}

\paragraph{Hyper-V} 
Je to hypervizorově stavený serverový systém pro 32 a 64 bitové systémy. Má svůj vlastní hlavní operační systém a pomocí virtualizace se skrze něj mohou spustit další operační systémy. Každá zádost o využití HW je převáděna pomocí tzv. VMBus (Virtual Machine Bus) na hlavní operační systém. Vyžaduje hardwarové akcelerace virtualizace (Intel VT nebo AMD-V) a NX kompatibilní procesor. \cite{WindowsInternalsPart1}

\begin{figure}[]
  \centering
  \includegraphics[height=7cm]{fig/hyperv_architektura.jpg}
  \caption{Architektura Hyper-V \cite{WindowsInternalsPart1}}
  \label{fig:hypervarchitektura}
\end{figure}

\paragraph{IIS} 
Internet Information Services je webový server s podporou rozšiřujících modulů. Podporuje protokoly HTTP, HTTPS, FTP, FTPS, SMTP a NNTP. Podporuje různé autentizační metody, např. Windows autentizace a klientský certifikát. Od verze 7.0 IIS má modulární architekturu.  Moduly, neboli rozšíření (extensions), mohou být přidány nebo odstraněny samostatně, nezávisle na sebe.
Existují různé typy modulů pro zajištění bezpečnosti, správu obsahu, komprese, cachování, logování a diagnostiku. Je dostupný rozšíření Web Deployment Tool, který umožňuje nasazení webových aplikací na IIS server. \cite{MasteringWindowsServer}

\paragraph{Storage Spaces} 
Technologie, která umožňuje virtualizovat úložiště dat pomocí seskupení disků do tzv. storage pools a pak vytvoření virtuálních disků nazývané storage spaces. Podporované možnosti jsou: Simple (bez duplikace), Mirror (zrcadlo s dvěma disků), Parity (distribuovaná parita, 3 disky). \cite{MasteringWindowsServer}

\paragraph{DHCP a DNS Server role} 
Role pro podporu protokolů DHCP a DNS. DHCP se používá pro automatickou konfiguraci počítačů připojených do sítě. DHCP server přiděluje počítačům IP adresu, masku sítě, implicitní bránu a adresu DNS serveru. DNS je hierarchický systém doménových jmen. Jeho hlavním úkolem je převod doménových jmen na IP adresy a naopak. \cite{MasteringWindowsServer}

\paragraph{Windows Update Services (WSUS)} 
WSUS je služba zajišťující aktualizaci softwaru pro operační systém Windows. Je to vlastně lokálně spravovaná podoba služby Windows Update. \cite{MasteringWindowsServer}

\paragraph{Remote Desktop Services (Terminal Services)} 
Pomocí této technologie je možné převzít kontrolu vzdáleného počítače. Na klienty je přenášeno jenom GUI rozhraní. Vstupy uživatele jsou přenášeny zpátky na server, kde jsou zpracovávany. \cite{MasteringWindowsServer}

\paragraph{Windows Deployment Services} 
Umožňuje vzdálené nasazení operačních systému Windows. Funguje na základě technologie PXE (Preboot Executioni Environment). \cite{MasteringWindowsServer}

\paragraph{Windows Remote Management}
WinRM je implementace protokolu WS-Management společností Microsoft. Protokol WS-Management je založen na základě protokolu SOAP (Simple Object Access Protocol). WS-Management a WinRM umožňují, aby různé hardwarové a softwarové komponenty vzájemně mezi sebou komunikovali. Na rozdíl od technologie DCOM (Distributed COM), WinRM nevyžaduje složitou konfiguraci firewallu. \cite{MasteringWindowsServer}

\paragraph{PowerShell Desired State Configuration}
PowerShell DSC je nová platforma ve Windows PowerShell. Umožňuje nasazení a správu konfigurace služeb a prostředí. DSC rozšiřuje jazyk Windows PowerShell. Poskytuje nové Windows PowerShell cmdlety a zdroje (resources), pomocí kterých je možné deklarativně popsat požadovaný stav a konfigurace. Příklady použití: zapínání/vypínaní role a funkce operačního systému, správa registrů, souborů a služeb, nasazení nového software, spuštění Windows PowerShell skriptů a zjištění aktuální konfigurace. \cite{dsc}

\subsection{Microsoft SQL Server}
SQL server je vysoce výkonný relační databázový systém založený na architektuře klient-server. Je škálovatelný a bezpečný a integrovaně pracuje s jazykem XML. 

SQL Server 2014 má tři edice:
\begin{itemize}
\item Enterprise: nejvyšší edice, jsou dostupné všechny funkce
\item Standard: limitované funkce Business Intelligence, typicky vhodné pro malé organizace
\item Business Intelligence: nabízí různé BI schopnosti -- reporting a analitics, Power View a Power Pivot. Vhodné pro organizace, které nepotřebují výkonný a škálovatelný OLTP, který je dostupný v edici Enterprise.
\end{itemize}

Dále jsou dostupné i jiné speciální edice: Developer, Web a Express.

Mezi hlavní komponenty patří: databázový stroj, reportování, služby pro analýzu a integraci dat, replikace, hlášení, fulltextové vyhledávání a Service Broker.

Nástroje dostupné pro správu: Microsoft SQL Management Studio, SQL Server Configuration Manager (konfigurace síťových protokolů), SQL Server Profiler (zachycování události ze serveru).  \cite{SqlServerIntro}

\subsection{TeamCity}
TeamCity je výkonný nástroj pro podporu Continuous Integration. Používá webové rozhraní pro správu. Podporuje celou škálou různých prostředí a platforem. Na obrázku \ref{fig:teamcityoverview} je vidět webové rozhraní aplikace\footnote{TeamCity demo je dostupné na adrese \url{https://teamcity.jetbrains.com/overview.html}}. Mezi hlavní výhody patří: 

\begin{itemize}
\item rozsáhlá podpora různých nástrojů: TeamCity má zabudovanou podporu pro velké množství systémů, např. Visual Studio, MSBuild, MSTest, GitHub, TFS, PowerShell, Bash, atd.
\item Rozsáhlá podpora konfigurace
\item Build History: funkce test history, change log, pinned builds, různé statistiky
\item Rozšiřitelnost: REST API, plugins, Third Party HTML reporty
\item Code Quality tracking: např. zabudovaný Code Coverage
\end{itemize}

Jsou dostupné i jiné nástroje (konkurence TeamCity): Jenkins/Hudson, TFS Build, Bamboo, CruiseControl, GitLab. \cite{TeamCity}

\begin{figure}[]
  \centering
  \includegraphics[height=10cm]{fig/teamcity_overview.jpg}
  \caption{Webové rozhraní v TeamCity}
  \label{fig:teamcityoverview}
\end{figure}

\subsection{Chef}
Chef je framework, pomocí kterého je možné jednoduše spravovat infrastrukturu. Chef je založen na skriptovacím jazyku Ruby. Umožňuje spravovat fyzické i virtuální servery. Bez zádrhelů vytvoříme dokonalé kopie hotových prostředí. Hlavní komponenty je vidět na obrázku \ref{fig:chefoverview}:

\begin{itemize}
\item Server: obsahuje konfigurace pro každý registrovaný uzel
\item Workstation: lokální Chef repository, tady pracuje vývojář
\item Node (uzel): klient, který je registrovaný na serveru. Má nainstalovaný agenta, který je nazván Chef client.
\item Chef Analytics: platforma, která poskytuje různé akce, historie, real-time reporty a notifikace.
\item Chef Supermarket: open source adresář, kde jsou dostupny cookbooky, které byly vyvinuty komunitou Chef.
\end{itemize}

Pojem \uv{cookbook} patří mezi nejdůležitější v nástroje Chef. Obsahuje konfigurace (tz. recipe), které musí být provedeny na uzlech. Každý cookbook definuje jeden kompletní scénář. \cite{ChefHandsOnLab}

\begin{figure}[]
  \centering
  \includegraphics[height=10cm]{fig/chef_overview.png}
  \caption{Přehled Chef}
  \label{fig:chefoverview}
\end{figure}

Konkurence Chefu je Puppet od firmy Puppet Labs. Výborné porovnání těchto nástrojů je dostupné na url \url{http://www.infoworld.com/article/2614204/data-center/puppet-or-chef--the-configuration-management-dilemma.html}. Hlavními výhodami Chefu jsou: jazyk Ruby (Puppet používá vlastní jazyk), zcela open-source bez omezení a podpora TDD (Test Driven Development) pomocí nástroje kitchen.

Obrázek \ref{fig:chefoverview} byl přebrán z webové stránky dostupné na \url{https://www.upguard.com/articles/puppet-vs.-chef-revisited}.

\subsection{Nástroje k řízení nasazovacího procesu}
\paragraph{HashiCorp Atlas}
Umožňuje rychlé nasazení pomocí automatizovaného pracovního postupu. Nasazovací proces rozšíří o možnost nastavení různých politik a pravidel. Atlas je velice flexibilní nástroj, ale i přesto poskytuje jednoduché a samoobslužné nasazování. \footnote{viz \url{https://www.hashicorp.com/atlas.html}}

\paragraph{Octopus Deploy}
Octopus Deploy velice dobře spolupracuje s buildovacím serverem, aby umožnil spolehlivé, bezpečné a automatizované nasazení ASP.NET webových aplikací a Windows služeb do testovacích a produkčních prostředí. Podporuje nasazení do cloudu, ale i do lokálních prostředí. \footnote{viz \url{https://octopus.com/}}

\paragraph{ThoughtWorks Go}
ThoughtWorks Go je open source continuous delivery server. Umožňuje vytvoření pokročilých deployment pipelinů. Placená podpora a různé rozšíření jsou taky dostupné, aby usnadnili komplikované scénáře. \footnote{viz \url{https://www.thoughtworks.com/go/}}

\paragraph{TFS Release Management}
Tento nástroj je součástí Microsoft TFS serveru. Umožňuje: automatizovat nasazení, odsouhlasení jednotlivých kroků, pokročilé logování, podporu bezpečnostních politik, snadné nasazení do Azure, rozsáhlou podporu rozšíření. \footnote{viz \url{https://www.visualstudio.com/en-us/features/release-management-vs.aspx}}

\paragraph{Chef Delivery}
Produkt od společnosti, která vytvořila i nástroj Chef. Jako ostatní nástroje, podporuje vytvoření pracovních postupů. Srdcem řešení je vlastně platforma Chef. \footnote{viz \url{https://www.chef.io/delivery}}

\subsection{Jiné technologie}
\paragraph{ASP.NET MVC 5}
ASP.NET MVC je webový aplikační framework, který implementuje vzor MVC (Model-View-Controller). Je to lehký, vysoce testovatelný framework, který je integrován s existujícímí funkcemi ASP.NET. \cite{aspnetmvc}

\paragraph{Application Insights}
Služba poskytovaná firmou Microsoft v cloudu Azure. Umožňuje monitorovat a diagnostikovat webové i mobilní aplikace, detekovat problémy a sledovat využití.\footnote{viz \url{https://azure.microsoft.com/en-us/services/application-insights}}

\paragraph{Power BI}
Je to sada business analytics nástrojů, pomocí kterého je možné analyzovat data a sdílet získané poznatky. Je to dostupná jako služba v kloudu Azure.\footnote{viz \url{https://powerbi.microsoft.com}}

\chapter{Praktická část}
\label{ch:impl}
V této kapitole se budeme zabývat s aplikací poznatků z předchozí kapitoly na naši ukázkovou aplikaci. Začínáme specifikací a návrhem aplikace, pak ukážeme implementaci jednotlivých částí.

\section{Úvod a specifikace}
Jako vývojářská firma jsme dostali úkol vytvořit aplikaci. Tato aplikace má být webovým informačním systémem, který eviduje peněžní transakce. Slouží jako interní nástroj, pomocí kterého lze analyzovat finanční toky. Důležitým požadavkem je, aby aplikace obsahovala i různé reporty. Některé transakce je potřeba elektronicky podepisovat a je vyžadováno integrace s existujícími systémy používanými zákazníkem. Domluvili jsme se, že budeme stavět na platformě společnosti Microsoft, tzn. .NET Framework a operační systém Windows.

Snažíme se používat agilní metodiky. Klademe veliký důraz na prototypování. Zákazník chce pravidelně vidět aktuální stav vývoje. Má možnost určit směr vývoje, prioritizovat vývoj jednotlivých funkcí a pravidelně kontrolovat, jestli produkt splňuje jeho očekávání. Pro nás nejsou tak důležité konkrétní a podrobné funkce aplikace, spíš to, že daná část aplikace vůbec existuje. 

Po analýze požadavku jsme identifikovali následující části:
\begin{itemize}
\item webová aplikace: klasická ASP.NET MVC 5 aplikace. Běží na aplikačním serveru IIS, v operačním systému Windows Server.
\item Microsoft SQL databáze: slouží jako úložiště dat v aplikaci.
\item klientské nástroje: prohlížeč neumí pracovat s různými integracemi a elektronickými podpisy, proto je potřeba, aby na klientských stanicích byly dostupné i desktopové aplikace. Komunikace mezi webovou stránkou a aplikací zajistí URI handlery.
\item reporty: Microsoft Reporting Services Report Server a kloudová služba Power BI.
\item scheduler: aplikace, která slouží k provádění pravidelných akcí.
\item monitoring provozu aplikace: monitoring aplikačního a databázového serveru, sledování výkonu serveru i aplikaci. Zvolená technologie je kloudová služba, Application Insights.
\end{itemize}

Na obrázku \ref{fig:architektura} vidíme základní architekturu aplikace. V následujících kapitolách ukážeme, jak splnit výše uvedené požadavky z hlediska nasazení. Aplikace bude nasaditelná již od začátku. Vycházíme z prázdného projektu a postupně přidáváme nové části. Budeme stavět komponenty, které lze využívat i pro jiné aplikace. Nemusíme nutně dodržet pořadí aplikování jednotlivých kroků a dokonce kroky můžeme provádět i paralelně.
 
\begin{figure}[]
  \centering
  \includegraphics[height=10cm]{fig/app_architektura.png}
  \caption{Architektura ukázkové aplikace}
  \label{fig:architektura}
\end{figure}

\section{Implementace}

\subsection{Webová aplikace}
Webové aplikace ASP.NET lze nasadit různými způsoby. Nejjednodušším z nich je pouhé kopírování souboru do určité složky na aplikačním serveru. Tento přístup má však svoje problémy. Neumí totiž upravit parametry webu (např. soubor \texttt{web.config}), ani aktualizovat nebo vytvořit webovou aplikaci v IIS. 

Další možností je využít nástroj Web Deploy. Tento nástroj je dostupný jako rozšíření do serveru IIS. Konfigurace spočívá ve vytvoření profilu nasazení, který je umístěn ve složce \texttt{<Projekt>/Properties/PublishProfiles}. Používá formát XML a je vlastně MSBuild skript. Část konfiguračního souboru pro projekt Nyx je přiložen v příloze \ref{ch:publishProfile}. Vytvoření balíčku lze pak zajistit pomocí příkazu:

\texttt{msbuild <solution.sln> /p:DeployOnBuild=true /p:PublishProfile=Deploy}

Výstupem je rovněž několik souborů: \texttt{Nyx.deploy.cmd}, \texttt{Nyx.SetParameters.xml}, \texttt{Nyx.SourceManifest.xml} a \texttt{Nyx.zip}. XML soubory slouží ke konfiguraci parametrů webu, \texttt{deploy.cmd} skript ke spuštění instalace a \texttt{Nyx.zip} obsahuje soubory webu.

Další parametry webu, jako např. konfigurace aplikačního poolu, vytvoření Web Sity, lze udělat pomocí nástroje \texttt{appcmd}, který je standardně součástí instalace IIS. Existuje i možnost použití PowerShell. Nejdříve je potřeba naimportovat příslušný modul: \texttt{PS> Add-PSSnapin WebAdministration} a pak použít cmdlety \texttt{New-WebAppPool}, \texttt{New-Website} a podobně.

Většina dnešních webových aplikací již běží nad protokolem HTTPS. Certifikát naimportujeme pomocí PowerShell (\texttt{Import-Certificate}). Další krok je navázání certifikátu na port 443. Toto lze udělat pomocí příkazu \texttt{netsh http add sslcert}. Příkazy jsou uvedeny v příloze \ref{ch:deployweb}.

Vytvořil jsem PowerShell modul \texttt{DeployWeb.psm1}, do kterého jsem zabalil výše uvedené funkce. Tento modul budeme využívat v rámci nasazení.

\subsection{Verzování databáze}
Zásady Continuous Integration musíme aplikovat nejen na zdrojový kód aplikace, ale i na databáze. To znamená, je potřeba 

\begin{itemize}
\item verzovat databázové objekty: tabulky, triggery, procedury, funkce atd.
\item vytvořit mechanizmus, pomocí kterého lze automatizovaně aktualizovat tyto objekty během nasazování. 
\end{itemize}

Nejdřív zajistíme ukládání verze databáze do jednoduché tabulky \texttt{DbVersion}. Tabulka se skládá ze tří sloupců: \texttt{Version}, \texttt{Note} a \texttt{Date}. Tato tabulka bude evidovat informace o aplikovaných verzí.

Verzování databáze ve Visual Studio lze udělat pomocí databázového projektu. V našem případě je to projekt \texttt{Nyx.DbProject}. Aktualizační mechanizmus, tz. mechanizmus k povýšení databáze, obsahuje projekt \texttt{Nyx.DbUp}. 

Úprava databázového schématu vypadá následovně:

\begin{enumerate}
\item Uděláme změny v databázi.
\item Pomocí funkce Schema Comparision (Tools $\rightarrow$ SQL Server $\rightarrow$ New Schema Comparision) provedeme porovnání databázového projektu a databáze.
\item Změny promítneme do databázového projektu.
\item Pomocí Schema Comparision lze nechat vygenerovat změnový SQL skript. Pokud potřebujeme, není problém tento skript napsat ručně.
\item Vytvořený změnový skript zaverzujeme do projektu \texttt{DbUp}. Existuje složka \texttt{ChangeScript}, která obsahuje tyto skripty. Tyto soubory jsou přímo vložené do výsledného exe souboru, proto musíme nastavit atribut \texttt{Build Action} na \texttt{Embedded Resource}.
\item Verze databáze zapíšeme i do souboru v databázovém projektu.
\end{enumerate}

Povýšení databáze pak provedeme spuštěním programu \texttt{dbup} z příkazové řádky.

Jak vidíme, projekt \texttt{Nyx.DbUp} obsahuje jenom databázové skripty. Funkcionalitu obsahuje projekt \texttt{Cadmus.DbUp}, který slouží jako znovupoužitelný prvek. Vychází ze knihovny \texttt{DbUp} (dostupný na \url{nuget.org} jako NuGet balíček). Udělal jsem několik úprav a doplnil různé parametry příkazové řádky, kontroly a logování.

Uvedený postup obsahuje mnoho manuálních kroků, tzn. je náchylné k chybám. Abychom proces zjednodušili a předešli chybám, bylo vytvořeno rozšíření do Visual Studia. Toto rozšíření umí nastavit atribut změnových skriptů \texttt{Build Action} na \texttt{Embedded Resource}, založit prázdné skripty a zapsat verze. Není to nezbytně nutná součást projektu, lze fungovat i bez ní. 

\subsection{Klientské nástroje}
Naše klientské nástroje jsou vlastně .NET aplikace. Tady se budeme zabývat s tím, jak tyto aplikace dostat na klientské stanice. Jednoduchým řešením je použít MSI instalátor. Takový instalátor ve Visual Studio dokážeme vytvořit mnoha způsoby. Já jsem se rozhodl pro nástroj WiX Toolset. Jde o projekt \texttt{Nyx.ClientTools.Setup}. Budeme potřebovat soubory aplikace nakopírovat a pak ještě nastavit URI handler v registrech. Hlavní konfigurační soubor, \texttt{Produkt.wsx}, je přiložen i jako příloha \ref{ch:wix}. 

Balíčky MSI lze nainstalovat z příkazové řádky pomocí nástroje \texttt{msiexec}. Příklad použití: \texttt{msiexec /i /qn Nyx.ClientTools.Setup.msi}, kde \texttt{/i} znamená instalace, \texttt{/qn} zajistí, aby nebylo zobrazeno žádné uživatelské rozhraní. Administrátor instalace může provádět pomocí Group Policy. Tento přístup má ale nevýhodu, že je potřeba, aby počítače byly v doméně a instalace bude provedena až po restartování počítače.

Jako alternativu ukážeme možnost instalace pomocí PowerShell Remoting. Nejdřív je potřeba funkce povolit. Nejjednodušším způsobem je spuštění PowerShell příkazu \texttt{Enable-PSRemoting}. Povolení lze nakonfigurovat i pomocí Group Policy, kde jednotlivé nastavení uděláme ručně (např. konfigurace firewallu, zapnutí servisy atd.). Instalace pak provádíme např. pomocí cmdletu \texttt{Invoke-Command}. Není problém instalaci spustit na více počítačích současně: názvy počítačů předáme jako parametr \texttt{-ComputerName}. 

\subsection{Testování}
Není naším cílem podrobně popsat různé testovací metody, jako unit testy, integrační testy atd., nicméně do určité míry se tady musíme zmínit o testování. Naše automatizované metody vyžadují, aby bylo možné jejich výstupy automatizovaně otestovat. 

Většina dnešních vývojářů je již zvyklý na použití unit testů. Dalším mohou být testy uživatelského rozhraní, které lze vytvořit pomocí nástroje Selenium.

Důležité je i otestovat skripty a nástroje konfigurace infrastruktury. Lze to zajistit pomocí nástroje \texttt{kitchen}, který je součástí Chef Development Kitu. Takový test pak ukážeme v kapitole \ref{ch:serverconfig}. V kapitole \ref{ch:ciserver} následně uvidíme integraci těchto nástrojů do Continuous Integration serveru. 

\subsection{Reporty}
report server, rs.exe, powerbi

\subsection{VM provisioning}
Packer, vagrant

\subsection{Server config management}
\label{ch:serverconfig}
chef, dsc

\subsection{CI server}
\label{ch:ciserver}
teamcity

\subsection{Blue/Green nasazení}
physical path, 2x web app

\subsection{Maintenance mód}
appoffline.htm

\subsection{Parametrizér a GUI}
xml, gui

\subsection{Orchestrátor nasazení}
powershell

\subsection{Release Management}
TFS Release Management

\subsection{Ukázkové prostředí}
Naše ukázkové prostředí se skládá z několika virtuálních strojů, které běží pod hypervisorem Hyper-V. Architekturu vidíme na obrázku \ref{fig:architekturasite}. 

\begin{figure}[]
  \centering
  \includegraphics[height=10cm]{fig/network.png}
  \caption{Architektura sítě}
  \label{fig:architekturasite}
\end{figure}

%Nástroje pro
%\begin{itemize}
%  \item konfigurace, parametre, nastavení + GUI
%  \item nasazovaní webu
%  \item konfigurace IIS
%  \item nasazovaní/vytvoření databáze
%  \item aktualizace databázového schématu
%  \item nasazení reportů (Report Server)
%  \item nasazení Schedulerů, použití Windows Task Scheduler
%  \item nasazení prerekvizit (.Net Framework, ...)
%  \item nastartování a údržba virtuálních prostředí
%  \item nástroje pro autentizace a předávání hesel (Windows, NTLM atd.)
%  \item deploy a výběr certifikátů
%  \item nástroje pro vývojáře: kopírování db, výběr prostředí, aplikování konfigů
%  \item install/system center
%  \item nástroje pro UI testování (Silverlight, HTML)
%  \item maintenance mód a podpora blue/green
%  \item instalace, konfigurace OS a virtuálních počítačů
%  \item deployment pipeline v TeamCity  
%  \item ...
%\end{itemize}

\chapter{Vyhodnocení výsledků}
\label{ch:vyslekdy}
V této kapitole se pokusíme vyhodnotit výsledky práce z předchozích kapitol. Budeme porovnávat nasazovací proces nejdříve bez, pak pomocí zásad Continous Delivery. 

\section{Nasazení bez CD}
Co to znamená, jak to vypadá, kolik to trvá. Jaké kroky, problémy, náchylnost k chybám...

\subsection{Popis}
Podrobný popis situace...

\subsection{Výhody, nevýhody}
Výhody jsou...
Nevýhody jsou...

\section{Nasazení pomocí CD}
Zavedli jsme zásady CD, jaké změny jsme uplatnili, v čem je to lepší, kolik to stálo, ...

\subsection{Popis}
Podrobný popis situace...

\subsection{Výhody, nevýhody}
Výhody jsou...
Nevýhody jsou...


\chapter{Závěr}
\label{ch:zaver}
Závěrem bude krátké shrnutí naší práce. Podařilo se to a to, nepodařilo se ....

%=========================================================================
