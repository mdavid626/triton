%==============================================================================
% (c) David Molnar 2015
%==============================================================================

\chapter{Úvod}
Vyvíjet software v dnešní době není jednoduché. Konkurence na trzích je silná, požadavky na software se mění velice rychle a často je potřeba vydávat a doručit funkční produkt co nejrychleji.

Jako základ pro vývojový tým je Continuous Integration. Pomocí této pomůcky tým zůstane v synchronizaci a dokáže odstranit zpoždění způsobené integračními chybami. Je důležité si ale uvědomit, že Continous Integration je pouze prvním krokem k dosažení cíle. Další stanice je Continuous Delivery, tzn. nejenom častá integrace, ale i časté nasazení a uvolnění software do produkce. Rozchodit nasazení ve smyslu Continuous Delivery není jednoduché, záleží to i na složitosti projektu. Nesmíme se ale zapomenout na okamžité výhody. Dlouhé, pracné a problematické releasování a nasazení se stane věcí minulosti. Díky tomu zákazníci uvidí okamžitý pokrok vývoje objednaného software, který dodá funkcionalitu, kterou potřebují a využívají každý den. Je potřeba se zmínit o výhodě, že tímto se také odstraní jeden u největších zdrojů stresu a napětí v týmu.

Velice inspirativní je projekt a tým Kenta Becka \cite{ContDelivery}. Tento velice disciplinovaný tým každý večer nasazoval novou verzi svého software-u přímo do produkce. Takový způsob implementace má několik výhod: napsaný a hotový kód neleží v SCM\footnote{Source Control Manager: SVN, Git, TFS atd.} bez využití, tým dokáže velice rychle reagovat na problémy a nové příležitosti. Navíc je zajímavý, že takový způsob práce vedlo k prohloubení vztahů mezi členy týmu a nárůstu důvěry zákazníka ve firmě. 

Continous Delivery dokáže snížit dobu trvání cyklu od první myšlenky a nápadu až do použitelný software. Myšlení ve smyslu Continuous Delivery bylo pro dlouhou dobu v zapomenutí někde mezi vývojovým a administrátorským týmem. Proto je velice důležité tyto dva týmy co nejvíce spojit. Nesmíme zapomenout na to, že základem všeho je vysoká stupeň automatizace -- operace budou rychlé, opakovatelné, snadno testovatelné a kvalitní (bez chyb).

Našim hlavním průvodcem v této práci bude Martin Fowler, průkopník a známý aktivista Continous Delivery. Jeho kniha \cite{ContDelivery}
bude pro nás sloužit jako Bible. 

Tato práce se zabývá zavedením konceptu Continuous Delivery, k podpoře rychlého vývoje a nasazení jedné ukázkové aplikace. Ukázkový projekt je webová aplikace s MSSQL databází. Dále obsahuje i jiné části, které budou popsané v kapitole xzy. 

V kapitole xzy je čtenář seznámen se základními pojmy z oblasti agilních metodik vývoje a doručení softwarových produktů. Kapitola xxx se detailně zabývá návrhem a implementací ukázkové aplikace. Vzhledem k tomu, že je to ukázková aplikace, nebudou všechny detaily implementovány. Pokračování v kapitole xxx bude popisovat vývoj jednotlivých nástrojů a vytvoření infrastruktury (Active Directory domain, testovací prostředí atd.), kam bude aplikace nasazená. Vyhodnocení výsledků se uskuteční v kapitole zyx. Pokusíme se porovnat výhody a nevýhody bez a s aplikováním Continous Delivery. Tato kapitola má sloužit k... V závěru bude následovat souhrn práce, ve které...

Našim cílem tedy bude ukázat, že je možný dodat vysoce kvalitní softwarový produkt velice často, třeba i denně a to bez zbytečných problémů a zádrhelů.

V celé této práci budeme označovat pojmy Continous Delivery zkratkou CD a Continous Integration zkratkou CI. Ostatní zkratky jsou ve tabulce zkratek na konci práce.

\chapter{Základy agilních metodik}
V této kapitole zavedeme čtenáře do agilních metodik a popíšeme koncept Continous Delivery.

Rozdíl mezi CD a CI.	

\section{Agilné metodiky}

\section{Continous Delivery}

\section{Nástroje}

\subsection{Windows Server}

\subsection{IIS}

\subsection{Microsoft SQL Server}

\subsection{TeamCity}

\subsection{Source Control}

\subsection{.NET Framework}

\chapter{Vývoj}

\section{Specifikace}

\section{Analýza}

\section{Implementace}


\chapter{Vyhodnocení výsledků}

\section{Vývoj a nasazení bez CD}

\subsection{Popis}

\subsection{Výhody, nevýhody}

\section{Vývoj a nasazení s CD}

\subsection{Popis}

\subsection{Výhody, nevýhody}


\chapter{Závěr}


%=========================================================================
